% ###########################
%
% snana-install Manual
% Started May 08, 2009
%
% Update history:
%
% date            comment
% Nov 02, 2010     include GSL 
% Mar 01, 2012     CERNLIB update from Rodney
% Mar 02, 2013     MINUIT_DIR and some general updating.
% Apr     2014     (ROOT and/or CERNLIB) or neither
%
% Feb 19  2015     minuit is part of snana, so don't need MINUIT_DIR
% Mar 10  2016     Fix text on how to compile w/wo HBOOK or ROOT
% Apr 05  2019     mention python >= 2.7 requirement
%
% ##########################


\documentclass[12pt]{article}
\usepackage{epsfig}          % use epsfig commands for graphs
\usepackage{url}

\setlength{\textheight}{8.5in}
\setlength{\textwidth}{6.8in}
\setlength{\oddsidemargin}{-0.2in}
\setlength{\topmargin}{+0.2in}
%\setlength{\footheight}{0.4in}
\setlength{\headsep}{0.0in}


\newcommand{\fitter}{{\tt snlc\_fit}}
\newcommand{\simu}{{\tt snlc\_sim}}
\newcommand{\mlcs}{{\tt MLCS2k2}}
\newcommand{\env}{environment}
\newcommand{\snana}{{\tt SNANA}}
\newcommand{\snanadir}{{\tt SNANA\_DIR}}
\newcommand{\sndataroot}{{\tt SNDATA\_ROOT}}
% \newcommand{\mndir}{{\tt MINUIT\_DIR}}
\newcommand{\mn}{{\sc minuit}}

\newcommand{\softdir}{{\tt \$SOFTDIR}}
\newcommand{\workdir}{{\tt \$SCRATCHDIR}}

\newcommand{\ROOT}{{\tt ROOT}}
\newcommand{\HBOOK}{{\tt HBOOK}}
\newcommand{\TEXT}{{\tt TEXT}}


% ========================================
\title{ \snana\ Installation Guide }

\author{J.~Bernstein, S.~Jha, R.~Kessler, S~Rodney }


\begin{document}

\maketitle
\tableofcontents

% -------------------------------------------------------
%                       START
% -------------------------------------------------------

% ####################################
   \section{Overview}
% ####################################


The installation of \snana\ involves two tarballs
available from the ``SNANA Download'' button.
First is the \snana\ software that includes all of
the source code and a {\tt Makefile}.
The second tarball, called ``\sndataroot,''
contains data (\S\ref{sec:data}), 
a simulation-output directory, 
and many files needed to run the \snana\ programs including
K-correction tables, model parameters,
filter responses, primary spectra (Vega, BD17), etc.
Environment variables 
\${\snanadir} and  \${\sndataroot}
must be defined to point to these areas,
and it is convenient to define these in a login script.
You will also need to add 
\$\snanadir/bin and \$\snanadir/util to your path.
It is recommended that \${\snanadir} be write-protected,
while \$\sndataroot\ has write-access for all users.
Once you have installed \snana, see the {\tt snana\_manual}
for instructions on running the programs.


To make the \snana\ binary executables,
\begin{verbatim}  
  cd $SNANA_DIR/src
  make
\end{verbatim}
% $
The \snana\ software is updated rather often, so it is recommended
to create an automated script to 'wget' the current \snana\ code
and run the make command.
The largest program size is under 400~Mb, but this does 
not include dynamic memory allocations.


% ####################################  
   \clearpage
   \section{Linux}
% ####################################


To use the \snana\ package you must install the 32-bit or 64-bit
versions of 
{\tt CFITSIO}, {\tt GSL} and {\tt libncurses}.
It is also recommended to install {\tt CERNLIB} and/or {\tt ROOT}
to make light curve plots and to access the full table of
analysis variables; however, neither of these packages is
required starting with {\tt v10\_34b}.
Finally, python 2.7 or higher is required.


In the following, \softdir\ refers to the top-directory
of your software packages,
and \workdir\ refers to your working area or scratch disk.
After installing the required packages and assuming you
are using the tcsh shell, add the following in your
login or {\snana}-setup script,
%
\begin{verbatim}
  > setenv CERN_DIR     $SOFTDIR/cern      ! optional (see below)
  > setenv ROOT_DIR     $SOFTDIR/root      ! optional (see below)
  > setenv CFITSIO_DIR  $SOFTDIR/cfitsio 
  > setenv GSL_DIR      $SOFTDIR/gsl
  > setenv PATH            $CERN_DIR/bin:$PATH 
  > setenv LD_LIBRARY_PATH $CERN_DIR/lib:$LD_LIBRARY_PATH 
  > setenv SNANA_DIR $SOFTDIR/SNANA/snana_v10_42i 
  > setenv PATH $SNANA_DIR/bin:$SNANA_DIR/util:$PATH
  > python --version  (must be 2.7 or higher)
\end{verbatim}
%$
and for bash make the apppropriate 'export' substitutions.


\noindent
Download the most recent \snana\ tarball to
{\softdir/\snana} and do
%
\begin{verbatim}
  > cd $SOFTDIR/SNANA
  > tar xzf snana_v10_42i.tar.gz 
  > cd snana_v10_42i/src 

  > make   
      (hold your breath ... )
\end{verbatim}
%
See \S\ref{sec:hbookroot} on how to enable or disable
compilation with \HBOOK\ and \ROOT.
Once you have the \snana\ sofware installed,
download the most recent {\sndataroot} tarball to 
{\workdir}/{\sndataroot}, and do
%
\begin{verbatim}
> setenv SNDATA_ROOT $SCRATCHDIR/SNDATA_ROOT
\end{verbatim}
%$
Based on the selection of HBOOK or ROOT,
the output file is selected in the namelist file with
\begin{verbatim}
   &SNLCINP
       HFILE_OUT    = 'anything.his'   ! HBOOK option
       ROOTFILE_OUT = 'anything.root'  ! ROOT  option
\end{verbatim}

% --------------------------------
\clearpage
\subsection{Troubleshooting}
% --------------------------------

\begin{enumerate}
\item 	If you get errors similar to 
\begin{verbatim}
$SOFTDIR/cern/2004/lib/libpacklib.a(cfclos.o)(.text+0xa): 
In function `cfclos_': : undefined reference to `rfio_close' 
%$
\end{verbatim}
%
  	then, in {\$\snanadir}/{\tt src/Makefile}, replace
  	``-lkernlib -lpacklib''  with
  	``-lkernlib\_noshift -lpacklib\_noshift''
%
  	if you have the latter with your {\tt CERNLIB} distribution. 
	If not, then try installing the appropriate version from the 
	{\tt CERNLIB} page. 
%
\item	If you have a 64-bit machine of type other than {\tt x86\_64}, 
   	then modify the BITNESS logical test in the {\tt Makefile} so 
   	that the  -m32 flag is used.
\end{enumerate}


% ###########################################
   \clearpage
   \section{Mac OS with Intel Processor}
% ##########################################

2014.05.27  Tested on Mac OSX v10.9 by S.Rodney.

\begin{enumerate}
\item	Make sure you have a working gfortran.\\
 Here are three possible sources:
  \begin{enumerate}
    \item RECOMMENDED: {\tt http://gcc.gnu.org/wiki/GFortranBinaries} : \\
     .dmg installer puts gfortran in {\tt /usr/local/gfortran/bin} \\
     add that to your {\tt \$PATH} or make sure links in {\tt /usr/local/bin} exist.
	 \item {\tt http://hpc.sourceforge.net}:\\
      unpack a tar ball, putting binaries into /usr/local/bin
    \item {\tt http://r.research.att.com/tools/} :\\
      .dmg installer puts binaries in /usr/local/bin\\
      they say "most other binaries are either incomplete or broken (do not use compilers from HPC, they won't work correctly!)."\\
      but that may be outdated.
   \end{enumerate}
%
%\item 	g95 is needed for compiling the Sussex cosmology fitter {\tt sncosmo\_mcmc}.
%   	For example, the binary from 
%	{\tt http://www.g95.org/downloads.shtml}.
%

\item 	Install SNANA dependencies : CFITSIO, GSL,
    and optionally {\tt ROOT} or {\tt CERNLIB}.

  There are two required dependencies for SNANA.  
  I recommend using a package installer, such as homebrew (\url{http://brew.sh}).  
  Then you simply need to be sure that the homebrew install location is on your PATH
  (it defaults to /usr/local, which should be fine).

\begin{verbatim}
# update homebrew itself, including recipes
brew update  

# upgrade the libraries
brew install gsl
brew install cfitsio
\end{verbatim}

Before upgrading SNANA in the future you can first upgrade these libraries using: 

\begin{verbatim}
brew update
brew upgrade gsl
brew upgrade cfitsio
\end{verbatim}

\item OPTIONAL : install ROOT 

Go to \url{http://root.cern.ch/} or pre-compiled libraries at \\
{\tt https://root.cern.ch/content/release-53434}


\item Be sure you have set the environment variables %% {\tt MINUIT\_DIR}, 
      {\tt ROOT\_DIR}, {\tt CFITSIO\_DIR}, and {\tt GSL\_DIR}, 
      following the instructions above. 
  
\item Build SNANA following the linux instructions above. 

\end{enumerate}


% ##########################################
%   \clearpage
  \section{Optional Compilation with \HBOOK\ and {\ROOT} }
  \label{sec:hbookroot}

\snana\ supports three output table formats:
\TEXT, \HBOOK, \ROOT.~~
\TEXT\ format requires no special libraries and is thus 
always available.
The \snana\ codes will compile with \HBOOK\ if 
environment (ENV) variable  {\tt\$CERN\_DIR} is defined.
\snana\ codes will compile with \ROOT\ if ENV {\tt\$ROOT\_DIR} 
is defined. 
To compile without \HBOOK\ or \ROOT, make sure that the 
associated ENV is not defined. Do not modify any files to 
adjust compilation with \HBOOK\ or \ROOT.


% ##########################################
%   \clearpage
   \section{Data Samples in Download}
   \label{sec:data}
% ##########################################

The downloads include SN data versions in 
\$\sndataroot/{\tt lcmerge}.\footnote{
``lcmerge'' refers to the merging of data and meta-data
such as PSF, skynoise, moon, etc ...}
Each SN data version corresponds to a published data set
that has been converted into the format needed
for the \snana\ light curve fitter.
To see a summary of the available data samples,
\begin{verbatim}
   > cd $SNDATA_ROOT/lcmerge/
   > ls */*.README
\end{verbatim}
% $
and then ``more'' any {\tt README} file for details.


% *******************************************************************
    \end{document}
% *******************************************************************

